% Emacs, this is -*-latex-*-

\title{\href{https://github.com/vicente-gonzalez-ruiz/vector_quantization}{Vector Quantization (VQ)}}

\maketitle
\tableofcontents

\section{Definition}
%{{{

In VQ, samples are quantized in groups (\emph{vectors}), producing a
quantization index by vector.  Usually, the lengths of the
quantization indexes are much shorter than the lengths of the vectors,
generating the compression. Vector Quantization (VQ) can remove
auto-correlation in the encoded signal and therefore, is more
efficient in RD terms than Scalar Quantization (SQ).

%Unfortunately, the
%computational requirements of VQ are, by far, much higher than the
%needed by SQ. If we also consider that there are other techniques
%(such as transform coding, that we will see later) that are able to
%decorrelate the samples requiring less computational resources than
%VQ, we can understand why SQ has been selected, for example, in most
%\href{https://en.wikipedia.org/wiki/Image_compression}{image} and
%\href{https://en.wikipedia.org/wiki/Video_coding_format}{video
%  codecs}.

%}}}

\section{TCQ}

Trellis-Coded Quantization (TCQ) process vectors of samples, using a
collection of non-overlapped scalar quantizers, and, depending on the
flexibility to chose one of these quantizers to encode the sample
${\mathbf s}_i$ (after having quantized the sample ${\mathbf
  s}_{i-1}$) we output a number of bits, for a given sequence of input
samples, that select the optimal order of quantizers that minimize the
distortion. Notice that the input is processed by blocks of samples,
and for this reason, some people refers TCQ as a type of
VQ. Therefore, TCQ inputs a block of samples and outputs a block of
bits (always with the same length). The advantage of TCQ is that the
output bits determine the order in which the quantizers must be used
to minimize the distortion encoding the input block.

The number of output bits depends on the number of quantizers and the
flexibility to select a quantizer\footnote{The higher the flexibility,
the higher the number of bits necessary to indicate the selected
quantizer.} after having used a previous one (that can be the
same). The transitions allowed between quantizers determine the so
called \emph{trellis}-step, that has so many steps as symbols are in
an input block. After building the complete trellis, the Viterbi
algorithm~\cite{} is used for selecting those sequence of quantizers
that minimize the distortion.\footnote{Basically, Viterbi found a way
of finding the most likely sequence of states of a finite-state
machine when the input (that describes the sequence) has been
corrupted by the noise. In our case, we want to find the sequence of
states (quantizers) that minimizes the distortion (quantization
error).}

\subsection*{Example}
Lets suppose the following (scalar and uniform) quantizer~\cite{}:

\begin{verbatim}
-3.5   -2.5   -1.5   -0.5    0.5    1.5    2.5    3.5
--+------+------+------+------+------+------+------+--
 000    001    010    011    100    101    110    111
\end{verbatim}

and that we want to quantize the input: 0.2, 1.6 and 2.3. If we use
directly such quantizer we get:

\begin{verbatim}
Input Output |Error| Code
-------------------------
  0.2    0.5     0.3  011
  1.6    1.5     0.1  101
  2.3    2.5     0.2  110
\end{verbatim}

The total error is (distortion) 0.6 and we need (rate) 9 bits.

Now lets use TCQ. Lets suppose that we have 4 quantizers:

\begin{verbatim}
-3.5   -2.5   -1.5   -0.5    0.5    1.5    2.5    3.5
--+------+------+------+------+------+------+------+--
  :      :      :      :      :      :      :      :
--+------:------:------:------+------:------:------:-- Q0 (00)
  0      :      :      :      1      :      :      :
---------+------:------:-------------+------:------:-- Q1 (01)
         0      :      :             1      :      :
----------------+------:--------------------+------:-- Q2 (10)
                0      :                    1      :
-----------------------+---------------------------+-- Q4 (11)
                       0                           1                 
\end{verbatim}
 
and that, for example:

\begin{verbatim}
FSM = [ [0, 2], [0, 2], [1, 3], [3, 1] ] # Finite State Machine
\end{verbatim}

Notice that \verb|FSM[0][0] = 0|, which means that we can use Q0 after
using Q0, encoding a 0, \verb|FSM[0][1] = 2|, which means that we can
use Q2 after using Q0, encoding a 1. Therefore, considering such FSM,
we need only to output one bit to determine the next used quantizer.

Now, lets encode:

\begin{verbatim}
Input     Q0       Q1       Q2         Q3
------+-----------------------------------
  0.2 |  0.3      1.3      1.7        0.7 <- |Quantization errors|, first sample
      |  | \     /   \    /   \      /  |
      | 0|   1\ /      \/      \ /      |
      |  |     /  \   /  \   /  \       | 
      |  |    /     / \  / \     \      |
      |  |   /    /   /  \   \    \     |
  1.6 | 0.3 1.3  1.7 0.7  0.3 1.3  1.7 0.7
      |   +   +    +   +    +   +    +   +
      | 1.1 0.1  0.1 0.1  0.9 0.9  1.9 1.9
      |   =   =    =   =    =   =    =   =
      | 1.4 1.4  1.8 0.8  1.2 2.2  3.6 2.6 <- |Accumulated quantization errors|, second sample
      |  | \     /   \    /   \      /  |
      |  |    \ /      \/      \ /      |
      |  |     /  \   /  \   /  \       | 
      |  |    /     / \  / \     \      |
      |  |   /    /   /  \   \    \     |
  2.3 | 1.4 0.8  1.2 2.6  1.4 0.8  1.2 2.6 
      |   +   +    +   +    +   +    +   +
      | 1.8 1.8  0.8 0.8  0.2 0.2  1.2 1.2
      |   =   =    =   =    =   =    =   =
      | 3.2 2.6  2.0 3.4  1.6 1.0  2.4 3.8 <- |Accumulated quantization errors|, third sample
\end{verbatim}

Therefore, the best combination of quantizers is (in reverse order)
Q2, Q1, Q3, generating the code-word <11><0><1> and a total distortion
of $1.0$.

\section{Matrix-based implementation}

\begin{verbatim}

Encoding sample 0.2:

             previous
        Q0    Q1    Q2    Q3
     +-----+-----+-----+-----+
  Q0 | 0.3 |     |     |     |
n    +-----+-----+-----+-----+
e Q1 |     | 1.3 |     |     |
x    +-----+-----+-----+-----+
t Q2 |     |     | 1.7 |     |
     +-----+-----+-----+-----+
  Q3 |     |     |     | 0.7 |
     +-----+-----+-----+-----+

Encoding sample 1.6:

             previous
        Q0    Q1    Q2    Q3
     +-----+-----+-----+-----+
  Q0 | 1.4 | 1.4 |     |     |
n    +-----+-----+-----+-----+
e Q1 |     |     | 1.8 | 0.8 |
x    +-----+-----+-----+-----+
t Q2 | 1.2 | 2.2 |     |     |
     +-----+-----+-----+-----+
  Q3 |     |     | 3.6 | 2.6 |
     +-----+-----+-----+-----+

Encoding sample 2.3:

             previous
        Q0    Q1    Q2    Q3
     +-----+-----+-----+-----+
  Q0 | 3.2 | 2.6 |     |     |
n    +-----+-----+-----+-----+
e Q1 |     |     | 2.0 | 3.4 |
x    +-----+-----+-----+-----+
t Q2 | 1.6 | 1.0 |     |     |
     +-----+-----+-----+-----+
  Q3 |     |     | 2.9 | 3.8 |
     +-----+-----+-----+-----+
\end{verbatim}

The minimum distortion is found at coordinate (Q2, Q1), which means
that the last quantizer is Q2, the previous one is Q1, and if we go to
row Q1 in the previous matrix, the minimum is found a column
Q3. Therefore, the optimal sequence of quantizers is Q3, Q1, Q2.

\section{Optimization of TCQ}
Notice that if the FSM is fully connected (and therefore, the trellis
too) the number of output bits will be higher and the distortion will
be lower. This generates an optimization problem in which we must find
the optimal FSM (number of quantizers and allowed transitions between
them), that should minimize the RD curve.

\section{Resources}

\renewcommand{\addcontentsline}[3]{} % Remove functionality of \addcontentsline
\bibliography{data-compression,signal-processing,DWT,image-processing}
